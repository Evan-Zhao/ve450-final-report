\subsection{Novelty and Main Contributions}
The novelty of our project is as follows:
\begin{enumerate}
    \item \textcolor{gray}{We use state-of-art deep learning tools for semi-automatic semantic segmentation, classification and human-pose detection for labeling of autonomous driving scenes. These methods are not usually used in other annotation software} \footnote{\textcolor{gray}{Response of Comment from DR3. Question: What does 'state-of-art' mean for your annotation tool? Answer: Most annotation software are either fully automatic or fully manumotive, ours is a semi-automatic one combining advantages from both. Besides, the methods we mentioned here are not used in others.}}.
    \item \textcolor{gray}{We give a thorough comparison of the different network architectures; for example, \emph{Yolov3}, \emph{deeplab}, \emph{Part Affinity Field} in the above tasks.}
    \item \textcolor{gray}{We increase the user-friendliness of the GUI by adding sub-tags such as occlusion/truncation tags. Other features will be also included as shown in tool functionalities mentioned above. }
    \item \textcolor{gray}{In terms of the models, we use testing-time augmentation (TTA), and incorporate intermediate layer heat map to help with labeling. }
    \item We do the online deployment of our project, which brings much convenience for users to do manual correction. Besides, the privacy of image sources will be effectively protected by the online version when compared with the local one. Because when outsourcing, we don't need to hand over the entire data set, but we can store the data in our backend database and restrict user access to these data.
    
\end{enumerate}
    
\subsection{Related Products}

\subsubsection{Labelme}
\textcolor{gray}{\textit{Labelme}\cite{labelme} is most popular semi-automated annotation tool for images on Github, and we decided to base our tool on it and make modifications. This tool supports several kinds of DNN models and different types of images, and provides good suggestion for annotation on normal images. However, it performs poorly on annotations for human gestures. It doesn't support sub-tagging and provides no color difference for different tags, which hinders managing a more reasonable and clearer structure of tags and makes users confused in complicated images. We will focus our modification on these points to provide a better user experience and higher efficiency. }


\subsubsection{Anno-Mage}

\textcolor{gray}{\textit{Anno-Mage}\cite{annomage} is a light-weight semi-automated annotation tool. It provides a simple and neat GUI with differently colored suggestions. But the DNN models it supports are not adequate for annotation for autonomous driving. It only supports 80 class objects from the \textit{MS COCO} dataset using a pretrained RetinaNet mode.}
