Our system mainly consists of two major parts: image preprocessing and GUI design. In the following we will describe the concepts generated in both parts. Fig.\ref{fig:MorphChart} is our morphological chart. Among listed functions, image augmentation, object detection, human pose estimation and semantic segmentation belongs to image preprocessing and the rest of them are the part of GUI design.
\begin{figure}[h!]
  \centering \includegraphics[width=\linewidth]{morp.png}
  \caption{Morphological chart}
  \label{fig:MorphChart}
\end{figure}

\subsection{Image Preprocessing}
We have done the literature research on several image processing methods. Combining our knowledge on basic machine learning and computer vision techniques, we come up with three concepts. On one hand, we consider traditional image processing methods to pre-annotate objects in images. On the other hand, machining learning offers more functionality and flexibility in preprocessing. In this category, we consider two concepts: traditional machine learning and deep learning. Fig. \ref{fig:backend} is our process of concept generation.
\begin{figure}[h!]
  \centering \includegraphics[width=\linewidth]{concept_backend.png}
  \caption{Concept generation of image preprocessing}
  \label{fig:backend}
\end{figure}


\subsection{GUI Design}
For GUI Design, we choose a popular open source GUI libraries, on which we build our software. Taking portability and code quality into account, after sifting through various libraries, we select Qt, WxWidgets and Tk as candidates.